\subsection{Scénarios d’utilisation}
\newcommand{\perso}{Robert}
\fbox{	\begin{minipage}{15cm}
		Pour le déroulement de cette partie nous allons suivre \perso, un ami des développeurs qui tente le jeu pour la première fois et qui n’a jamais joué à un jeu de réflexion au-paravant.\\

		\perso, obtient les fichiers sources du jeu et lance le \executable contenu dans les dossiers.\\
		La page d'accueil se lance sans problème, \perso clique sur \textcolor{blue}{jouer}.
		Plusieurs modes de jeu lui sont alors proposés :
		\begin{itemize}
			\item Continuer la campagne
			\item Démarrer une nouvelle campagne
			\item Faire une partie rapide
		\end{itemize}
		Il choisit de faire une partie rapide, afin de pouvoir tester le jeu.\\
		Alors, il lui est proposé de choisir la difficulté, de voir plus d'option ou de lancer la partie.\\
		\perso choisit la difficulte \textcolor{blue}{facile} vu que c'est ça première partie.
		Il lance le jeu sans même regardé les autres options disponible.\\

		Il arrive alors sur une grille de kakuro, de 15*15 cases.\\
		Malheureusement \perso se rend compte, une fois devant sa grille, qu’il ne connait même pas les règles.
		Il décide donc de retourner sur le menu principal, car, il se souvient avoir vu "aide".

		Pour ce faire, il clique sur le bouton avec une fléche allant vers une porte entre-baillé.\\
		Il clique sur le bouton \textcolor{blue}{aide} qui l'intéresser et se retrouve sur un nouveau menu :
		\begin{itemize}
			\item Voir les règle du kakuro
			\item Comment utilisé l'interface de jeu
			\item Truc et astuce du kakuro
			\item Explication des modes de jeu
			\item Retour à l'accueil
		\end{itemize}

		Après avoir pris connaissance des règles, \perso décide de retourné jouer.\\
		Lorsque \perso clique sur le bouton \textcolor{blue}{jouer} du menu principal, il se retrouve face au menu de selection de modes de jeu qu'il connait déjà.\\
		Cepandant, une nouvelle option se présente maintenant devant lui : \textcolor{blue}{Reprendre la dérnière partie}.
		Il clique sur ce bouton, et se retrouve sur la partie qu'il avait déjà commencé un peu plus tôt.\\

		Après avoir fini de jouer, \perso, qui était sur une partie rapide est renvoyé à l'acceuil, où il clique sur \textcolor{blue}{classement} où il voit, trônant fièrement son nom, comme seul score jamais enregistrer (et donc premier du classement).
\end{minipage}	}
